% --------------------------------------------------------------
% This is all preamble stuff that you don't have to worry about.
% Head down to where it says "Start here"
% --------------------------------------------------------------

%Preamble
\documentclass[12pt]{article}
\usepackage[margin=1in]{geometry} 
\usepackage{amsmath,amsthm,amssymb,scrextend,mathtools}
\usepackage{tikz}
\usetikzlibrary{arrows}
\usetikzlibrary{positioning}
\usepackage{fancyhdr}
\usepackage{forest}
\pagestyle{fancy}

%Command Declarations
\newcommand{\N}{\mathbb{N}}
\newcommand{\Z}{\mathbb{Z}}
\newcommand{\I}{\mathbb{I}}
\newcommand{\R}{\mathbb{R}}
\newcommand{\Q}{\mathbb{Q}}
\renewcommand{\qed}{\hfill$\blacksquare$}
\let\newproof\proof
\renewenvironment{proof}{\begin{addmargin}[1em]{0em}\begin{newproof}}{\end{newproof}\end{addmargin}\qed}

%Environment Declarations 
\newenvironment{theorem}[2][Theorem]{\begin{trivlist}
\item[\hskip \labelsep {\bfseries #1}\hskip \labelsep {\bfseries #2.}]}{\end{trivlist}}
\newenvironment{lemma}[2][Lemma]{\begin{trivlist}
\item[\hskip \labelsep {\bfseries #1}\hskip \labelsep {\bfseries #2.}]}{\end{trivlist}}
\newenvironment{problem}[2][Problem]{\begin{trivlist}
\item[\hskip \labelsep {\bfseries #1}\hskip \labelsep {\bfseries #2.}]}{\end{trivlist}}
\newenvironment{exercise}[2][Exercise]{\begin{trivlist}
\item[\hskip \labelsep {\bfseries #1}\hskip \labelsep {\bfseries #2.}]}{\end{trivlist}}
\newenvironment{reflection}[2][Reflection]{\begin{trivlist}
\item[\hskip \labelsep {\bfseries #1}\hskip \labelsep {\bfseries #2.}]}{\end{trivlist}}
\newenvironment{proposition}[2][Proposition]{\begin{trivlist}
\item[\hskip \labelsep {\bfseries #1}\hskip \labelsep {\bfseries #2.}]}{\end{trivlist}}
\newenvironment{corollary}[2][Corollary]{\begin{trivlist}
\item[\hskip \labelsep {\bfseries #1}\hskip \labelsep {\bfseries #2.}]}{\end{trivlist}}
\newenvironment{definition}[2][Definition]{\begin{trivlist}
\item[\hskip \labelsep {\bfseries #1}\hskip \labelsep {\bfseries #2.}]}{\end{trivlist}}
 
\begin{document}
 
% --------------------------------------------------------------
%                         Start here
% --------------------------------------------------------------

%Header
\lhead{CSCI 2270: Assignment 7}
\chead{Justin Adkins}
\rhead{March 16, 2018}
\textbf{Question 1:} Does inserting a node into a red-black tree, re-balancing, and then deleting it result in the original tree? \\ \\
\noindent
We will show that inserting, rebalancing, and then deleting a node \emph{can} result in a new tree. Consider the following tree: \\
\tikzset{
  treenode/.style = {align=center, inner sep=0pt, text centered,
    font=\sffamily},
  arn_n/.style = {treenode, circle, white, font=\sffamily\bfseries, draw=black,
    fill=black, text width=1.5em},% arbre rouge noir, noeud noir
  arn_r/.style = {treenode, circle, red, draw=red, 
    text width=1.5em, very thick},% arbre rouge noir, noeud rouge
  arn_x/.style = {treenode, rectangle, draw=black, fill = black,
    minimum width=0.5em, minimum height=0.5em}% arbre rouge noir, nil
}
\begin{center}
\begin{tikzpicture}[->,>=stealth',level/.style={sibling distance = 4cm/#1,
  level distance = 1cm}] 
\node [arn_n] {10}
    child{ node [arn_x] {}
	   }
    child{ node[arn_r] {20}
}; 
\end{tikzpicture}
\end{center}

\noindent
Adding 30 to this tree will result in the following: \\

\begin{center}
\begin{tikzpicture}[->,>=stealth',level/.style={sibling distance = 4cm/#1,
  level distance = 1cm}] 
\node [arn_n] {10}
    child{ node [arn_x] {}
	   }
    child{ node[arn_r] {20}
            child{ node [arn_x] {} }
    	    child{ node [arn_r] {30} }
}; 
\end{tikzpicture}
\end{center}
\noindent
This tree is not balanced because it contains a red node with a red child. From the RBInsert Algorithm line 11, we perform a left rotate on 20 (parent of 30). This will make 20 the root, and by line 17, the root will be colored black. This results in this tree: \\

\begin{center}
\begin{tikzpicture}[->,>=stealth',level/.style={sibling distance = 4cm/#1,
  level distance = 1cm}] 
\node [arn_n] {20}
    child{ node [arn_r] {10}
	   }
    child{ node[arn_r] {30}
}; 
\end{tikzpicture}
\end{center}
\noindent
Now deleting 30 from the tree we are left with a tree that is different from the original:

\begin{center}
\begin{tikzpicture}[->,>=stealth',level/.style={sibling distance = 4cm/#1,
  level distance = 1cm}] 
\node [arn_n] {20}
    child{ node [arn_r] {10}
	   }
    child{ node[arn_x] {}
}; 
\end{tikzpicture}
\end{center}

\textbf{Question 2:} Does deleting a node with no children from a red-black tree, re-balancing, and then re-inserting it with the same key always result in the original tree? \\

\noindent
We will show that by deleting a node with no children, re-balancing, and then re-inserting the same node can result in a different tree. Consider the following tree: \\

\begin{center}
\begin{tikzpicture}[->,>=stealth',level/.style={sibling distance = 4cm/#1,
  level distance = 1cm}] 
\node [arn_n] {20}
    child{ node [arn_n] {10}
	   }
    child{ node[arn_n] {30}
    	child{ node[arn_x] {}}
    	child{ node[arn_r] {40}}
}; 
\end{tikzpicture}
\end{center}
\noindent
Deleting 10 will result in an unbalanced tree. The RBBalance Algorithm will be called since $x$ will be a null node black child of 20. This will result in a left rotate on 20 (line 21 of RBBalance) and the re-coloring of 40 (line 20 of RBBalance). The resulting tree is: \\
\begin{center}
\begin{tikzpicture}[->,>=stealth',level/.style={sibling distance = 4cm/#1,
  level distance = 1cm}] 
\node [arn_n] {30}
    child{ node [arn_n] {20}
	   }
    child{ node[arn_n] {40}
}; 
\end{tikzpicture}
\end{center}
\noindent
Now, re-inserting the 10 node we get the following: \\
\begin{center}
\begin{tikzpicture}[->,>=stealth',level/.style={sibling distance = 4cm/#1,
  level distance = 1cm}] 
\node [arn_n] {30}
    child{ node [arn_n] {20}
    	child{ node [arn_r] {10}}
    	child{ node [arn_x] {}}
	   }
    child{ node[arn_n] {40}
}; 
\end{tikzpicture}
\end{center}
\noindent
This tree needs no rebalancing after insertion and is a valid Red Black Tree. Comparing it to our original tree, we can see that the two differ in several areas. Thus, proving that deleting a node with no children, re-balancing, and re-inserting the same node \emph{can} result in a different tree.
\end{document}









